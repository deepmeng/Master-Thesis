\chapter{Introduction}\label{Introduction}
With the advances in the computational power of processors, new opportunities for growth opened up in the world. A lot of algorithms that are well known for decades couldn't be tested because powerful computational possibilities were not available at the time of their invention. Nowadays this is not a problem anymore. The expansion in computational power impacted the bloom of a very impressive domain as artificial intelligence that, not long ago, has been only a product of science fiction movies. The different machine learning algorithms that were once just mathematical models, are now implemented in real world applications that make wonders. This rapidly growing field has already made a big difference in people's every day lives through applications like Google assistant and Google self-driving car - Waymo.

Among the various paradigms in machine learning, a very notable one is reinforcement learning, which has been especially paid attention to in the most recent years and proven to make incredible results. Reinforcement learning is a known field in cognitive science that focuses on the study of the thinking processes. The main idea is that learning happens with the help of a feedback coming from the outer world and this exact idea is resembled in the reinforcement learning algorithms. 

In psychology, the idea of trial-and-error has been expressed by Edward Thorndike in 1911 \cite{Sutton:1998:IRL:551283}. So, this very idea existed for a long time until it could become the basis of a machine learning paradigm. Actually, the thought about applying it in computers came to life as soon as the thought of computers appeared in the Turing period, in 1950 \cite{Sutton:1998:IRL:551283}. In it's development it was first mixed up with supervised learning, so that later after the 70s to become more and more refined. Amidst the most representative books is the one written by Richard S. Sutton and Andrew G. Barto called "Introduction to Reinforcement Learning" \cite{Sutton:1998:IRL:551283} originally published in 1998. The book \cite{Sutton:1998:IRL:551283} provides a strong theoretical foundation and it's currently being worked on for a second edition \cite{Sutton}.

Lately, DeepMind made important contributions in artificial intelligence by also using reinforcement learning algorithms together with deep learning techniques in neural networks. They have produced some state-of-the art papers that became truly meaningful for the technological advancement nowadays. These include the papers "Playing Atari with Deep Reinforcement Learning" \cite{DBLP:journals/corr/MnihKSGAWR13} - 2013, where deep Q-learning is used, and the biggest most recent breakthrough, "Mastering the Game of Go with Deep Neural Networks and Tree Search" \cite{Silver_2016} - 2016, where the reinforcement learning Monte Carlo tree search is used with Q-learning. Also, in June, 2016, they publish a paper "Asynchronous Methods for Deep Reinforcement Learning" \cite{DBLP:journals/corr/MnihBMGLHSK16} that provides a clear comparison of the different deep reinforcement learning algorithms for the asynchronous training emphasizing an algorithm - "Asynchronous Advantage Actor Critic" that surpasses the Atari performances.

In the abundant flow of discoveries in artificial intelligence and deep reinforcement learning, the reuse of the new techniques and their further exploration was imminent. This has also been the case of this project.

\textbf{Scope}

\textbf{Structure}
